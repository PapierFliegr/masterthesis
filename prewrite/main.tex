\documentclass[11pt]{article}

% ---------------------------- packages ---------------------------- %
\usepackage{amsmath, mathtools, amsfonts, amsthm} % math symbols
\usepackage{import}
\usepackage{enumitem} % enumerate
\usepackage{hyperref} % hyperlinks
\usepackage{cleveref} % cref

% -------------------------- new theorems -------------------------- %
\newtheorem{definition}{Definition}[section]
\newtheorem{theorem}[definition]{Theorem}
\newtheorem*{note}{Note}

% ---------------------------- commands ---------------------------- %
\newcommand*{\dd}[1]{\mathop{}\!\mathrm{d}#1} % dx Integral
\newcommand*{\ddd}[2]{\mathop{}\!\mathrm{d^{#1}}#2} % dx higher dim

% ---------------------------- document ---------------------------- %
\begin{document}


\section{Figure of a graph}
\label{sec:001}

huhfiusdbf iuhfui sdhfuihsduf sdiuhsd fiusdhf dsfiusdhf suisdfh dshiusdhf sdiuhsdf uidiu
fdsf sdfsdknf fds oihfiwuehf udshfuidshf uidhf usdhf dshfisdufh hfds fiusdhf uihfu hsuifh
iusdhf uisdhf sdhuifhsdiuhfusdhf uhiufhuisdhf uihsduifh suihfusdhfuh iushdfuihsd
uifhsduifsd fhsdiuf hsduifh uisdhuihsuidhfiu shfuihsdiu fhsdiufh sdifhsdiu fuisdhf hiu
sduif sduifh dsfuidshf sdiufh iusdfhiusd fisudfh dsufihsdiuf sduifhsdui fhdsuifhsdui
fsduifh sdiufh sdiufhsdiufh

\begin{figure}[h]
	\centering
	\def\svgwidth{0.4\textwidth}
	\import{figures}{drawing.tex.pdf_tex}
	\caption{Test}
\end{figure}

huhfiusdbf iuhfui sdhfuihsduf sdiuhsd fiusdhf dsfiusdhf suisdfh dshiusdhf sdiuhsdf uidiu
fdsf sdfsdknf fds oihfiwuehf udshfuidshf uidhf usdhf dshfisdufh hfds fiusdhf uihfu hsuifh
iusdhf uisdhf sdhuifhsdiuhfusdhf uhiufhuisdhf uihsduifh suihfusdhfuh iushdfuihsd
uifhsduifsd fhsdiuf hsduifh uisdhuihsuidhfiu shfuihsdiu fhsdiufh sdifhsdiu fuisdhf hiu
sduif sduifh dsfuidshf sdiufh iusdfhiusd fisudfh dsufihsdiuf sduifhsdui fhdsuifhsdui
fsduifh sdiufh sdiufhsdiufh

\section{Result 01}
\label{sec:002}

\begin{theorem}[Model 01: connected]
	\label{thm:001}
	Let \( \Omega \coloneqq \{(x^\prime, x_n) \text{ s.t. }\lvert x^\prime \rvert
	\leq 1,\lvert x_n \rvert \leq M \} \) and \( E_0 \coloneqq \{ (x^\prime,x_n)
	\text{ s.t. } \lvert x^\prime \rvert \leq R, \lvert x_n \rvert \geq M \} \) for
	some \( R, M > 0 \). Then there exists an \( M_0 \) s.t.\ for all \( M \leq M_0 \)
	the minimizer of the fractional perimeter is connected and given by \( E_M = \Omega
	\cup E_0 \).
\end{theorem}

\begin{proof}
	Proof by contradiction. Assume \( E_M \) is not \( E_0 \cup \Omega \), then we
	can slide a ball of radius \( r \) down and at some point it will touch \( E_M \).
	We consider the ball \( B_r (t e_n) \). Since \( E_M \) not cylindrical, there
	exists \( r_0 \in (0,1) \) and \( t_0 > 0 \) s.t.\ \( \partial B_{r_0}(t_0
	e_n) \cap \partial E_M \neq \emptyset \) and \( B_{r_0}(t e_n) \subset E_M
	\) for all \( t > t_0 \). \\
	Since \( E_M \) is a minimizer it is also a variational solution and the
	inequality holds
	\[
		\int_{\mathbb{R}^n} \frac{\chi_{E_M^c}(y)-\chi_{E_M} (y)}{\lvert y-q\rvert^{n+s}} \dd{y} \geq 0
	\]
	whereas \( q \in \partial B_{r_0}(t_0 e_n) \cap \partial E_M \). \\
	We show that the left hand side is negative. Split the domain into four parts, as seen
	in the Figure:
	\begin{figure}[h]
		\centering
		\includegraphics[width = 0.5\textwidth]{figures/Screenshot_20240105_020915.png}
		\caption{}
		\label{fig:001}
	\end{figure}
	\par
	We define
	\begin{align*}
		A                                & \coloneqq \{ (x^\prime,x_n) \text{ s.t. } \lvert x^\prime -q^\prime \rvert \geq R+1\} \text{ Green Area} \\
		B                                & \coloneqq \{ (x^\prime,x_n) \text{ s.t. } \lvert x^\prime \rvert < R, \lvert x_n -q_n \rvert > 2M \} \\
		C                                & \coloneqq \{ (x^\prime,x_n) \text{ s.t. } \lvert x^\prime \rvert \geq R, \lvert x^\prime - q^\prime \rvert \leq R+1, \lvert x_n -q_n \rvert > \Lambda M \} \\
		\text{Everything else} \subset S & \coloneqq \{(x^\prime,x_n) \text{ s.t. } \lvert x^\prime -q^\prime \rvert \leq R+1, \lvert x_n -q_n \rvert \leq \Lambda M \}
	\end{align*}


	Integration over the first part:
	\begin{gather*}
		\int_A \frac{\chi_{E^c} -\chi_E}{\lvert y-q\rvert^{n+s}} \dd{y} \overset{A \subset E^c}{ =} \int_{ \lvert y^\prime \rvert > R+1} \frac{1}{\lvert y \rvert^{n+s}} \dd{y} \leq c(n) \int_{R+1}^\infty r^{-s-2} \dd{y} \leq c(n,s) R^{-(1+s)}
	\end{gather*}

	Integration over the second part:
	\begin{gather*}
		\int_B \frac{\chi_{E^c} -\chi_E}{\lvert y-q \rvert^{n+s}} \dd{y} \overset{B \subset E}{ =} - \int_B \frac{1}{\lvert y-q\rvert^{n+s}} \dd{y} \leq -c(n,s) M^{-s} \qquad \text{Idea: Consider ball with factor \( 2^{-n} \)}
	\end{gather*}

	Integration over the third part:
	\begin{align*}
		 & \int_C \frac{\chi_{E^c} -\chi_E}{\lvert y-q\rvert^{n+s}} \dd{y} \overset{C \subset E^C}{ =} \int_C \frac{1}{\lvert y-q\rvert^{n+s}} \dd{y} \leq c(n) \int_{R-1}^{R+1} \int_{\Lambda M}^\infty \frac{r^{n-2}}{(r^2 +y_n^2)^{\frac{n+s}{2}}} \dd{y_n} \dd{r} \\
		 & \overset{r^2 \leq r^{2 + y_n^2}}{ \leq} c(n) \int_{R-1}^{R+1} \int_{2
		\Lambda M}^\infty \frac{1}{(r^2 +y_n^2)^{\frac{s+2}{2}}} \dd{y_n} \dd{r} \overset{\text{convexity}}{\leq} \int_{R-1}^{R+1} \int_{\Lambda M}^\infty \frac{1}{(r+y_n)^{s+2}} \dd{y_n} \dd{r} \\
		 & \leq c(n,s) \int_{R-1}^{R+1} \frac{1}{(r+\Lambda M)^{s+1}} \leq c(n,s)(R-1+\Lambda M)^{-s} \leq c(n,s)(\Lambda M)^{-s}
	\end{align*}

	Integration over the fourth part: \\
	Justifiction that we can estimate with \( S \): Only negative part of the integration
	is fully in the set we want to estimate and the rest in \( S \) is positive. \\
	We split \( S \) into four parts:
	\begin{enumerate}[label = \roman*)]
		\item \( S \cap B_{\Lambda M} (q) \cap B_{r_0}(z) \)
		\item \( S \cap B_{\Lambda M} (q) \cap B_{r_0}( \overline{z}) \)
		\item \( S \cap (B_{\Lambda M} (q)\setminus ( B_{r_0}(z) \cup B_{r_0}(\overline{z}))) \)
		\item \( S \setminus B_{\Lambda M} (q) \)
	\end{enumerate}
	where \( \overline{z}\coloneqq z + 2(q-z) \) and \( \Lambda > 4 \) chosen big enough
	and \( M \) chosen small enough s.t.\ \( \Lambda M \leq 1 \). \\
	We estimate the first and second part:
	\begin{align*}
		     & \int_{S \cap B_{\Lambda M} (q) \cap B_{r_0}(z) \cup S \cap B_{\Lambda M} (q) \cap B_{r_0}(\overline{z})} \frac{\chi_{E^c}- \chi_E}{\lvert y-q\rvert^{n+s}} \dd{y} \\
		\leq & \int_{S \cap B_{\Lambda M} (q) \cap B_{r_0}(z)} \frac{1}{\lvert y-q\rvert^{n+s}} \dd{y} - \int_{S \cap B_{\Lambda M} (q) \cap B_{r_0}(\overline{z})} \frac{1}{\lvert y-q\rvert^{n+s}} \dd{y} \leq 0
	\end{align*}
	These two intgerals cancel because of symmetry. \\
	We estimate the third part:
	\begin{gather*}
		\int_{S \cap (B_{\Lambda M} (q)\setminus ( B_{r_0}(z) \cup B_{r_0}(z)))}\frac{\chi_{E^c}- \chi_E}{\lvert y-q\rvert^{n+s}} \dd{y} \leq \int_{P_{1,\Lambda M}} \frac{1}{\lvert y-q\rvert^{n+s}} \dd{y} \leq C \Lambda^{1-s} M^{1-s}
	\end{gather*}
	where we used lemma 3.1 in 2016 dipierro-savin-valdinoci with \( R = r_0 = 1 and \lambda
	= \Lambda M \) (we can choose \( r_0 = 1 \), since if we can show the bound for \(
	r_0 = 1 \) then it holds for all smaller balls as well). \\
	We estimate the fourth part:
	\begin{gather*}
		\int_{S\setminus B_{\Lambda M} (q)} \frac{\chi_{E^c}- \chi_E}{\lvert y-q\rvert^{n+s}} \dd{y} \leq \int_{B_{R+2}\setminus B_{\Lambda M}} \frac{1}{\lvert y\rvert^{n+s}} \dd{y} = c(n,s)((\Lambda M)^{-s} - (R+2)^{-s})
	\end{gather*}
	since \( S \subset B_{R+2} \) for \( R \geq 1 \) since \( ((\Lambda M)^2 +
	(R+1)^2)^{\frac{1}{2}} \leq (R^2 + 4R+4)^{\frac{1}{2}} = R+2 \). \\
	\par
	Thus in total we get:
	\begin{align*}
		\int_{\mathbb{R}^n} \frac{\chi_{E^c}- \chi_E}{\lvert y-q\rvert^{n+s}} \dd{y}
		 & \leq -c_1 M^{-s} + c_0 (R^{-(1+s)} + (\Lambda M)^{-s} + (\Lambda M)^{-s} - (R+2)^{-s} + \Lambda^{1-s} M^{1-s}) \\
		 & \leq -c_1 M^{-s}(1- + \frac{c_0}{c_1} (R^{-(1+s)}M^s + 2 \Lambda^{-s} -(R+2)^{-s} M^s + \Lambda^{1-s} M
		\intertext{Choose \( \Lambda \) large and \( M \) small enoguh}
		 & \leq -c_2 M^{-s} < 0
	\end{align*}
\end{proof}


\section{Result 02}
\label{sec:003}

\begin{theorem}[Model 01: disconnected]
	\label{thm:002}
	Let the setting be as in theorem \cref{thm:001}, then there exists an \( M_0 \)
	s.t.\ for all \( M \geq M_0 \) the minimizer of the fractional perimeter is disconnected.
\end{theorem}
\begin{proof}
	Proof analogous to theorem 1.2 in 2016 dipierro-onoue-valdinoci. \\
\end{proof}

\section{Result 03}
\label{sec:004}

\begin{theorem}[Model 02: connected]
	\label{thm:003}
	Let \( \Omega \coloneqq \{(x^\prime, x_n) \text{ s.t. } \lvert x^\prime \rvert \leq
	1, \lvert x_n \rvert \leq M \} \) and \( E_0 \coloneqq \{ (x^\prime, x_n) \text{
		s.t. } M \leq \lvert x_n \rvert \leq R+M \} \). For every \( R > 0 \) there exists a
	\( M_0 \) s.t.\ for all \( M \leq M_0 \) the minimizer of the fractional perimeter
	is connected. (We need \( R > M \))
\end{theorem}
\begin{note}
	To prove this we need to show that minimizers are always connected to \( \Omega^c
	\) and i.e.\ \( d(E_0, \Omega) = d(E\setminus E_0, \Omega) \).
\end{note}
\begin{proof}
	We show that for every \( R > 0 \) at least the tube \( \{ \lvert x_n \rvert < r_0 \}
	\) is in the minimizer for some \( r_0 > 0 \). \\
	We do that analogously to theorem \cref{thm:001} by contradiction. We assume that \(
	E_M \) is disconnected, thus we can slide a ball of radius \( r \) down and for all
	\( r_0 \in (0,1) \) there exists a \( t_0 > 0 \) s.t.\ \( \partial B_{r_0}(t_0 e_n
	) \cap \partial E_M \neq \emptyset \). If we can show that there exists a \( r_0
	\) s.t.\ this conntradicts then the tube is in the minimizer. It is enough to show
	that for one \( r_0 \) since if we can contradict this for one \( r_0 \) then for
	all smaller there is no touching as well. \\
	For that we split into four parts as seen in the figure:
	\begin{figure}[h]
		\centering
		\includegraphics[width = 0.8\textwidth]{figures/Screenshot_20240107_160324.png}
		\caption{}
		\label{fig:002}
	\end{figure}
	\par
	We define
	\begin{align*}
		A & \coloneqq \{(x^\prime,x_n) \text{ s.t. } \lvert x_n \rvert \geq M+R \} \\
		B & \coloneqq \{(x^\prime,x_n) \text{ s.t. } \lvert x_n \rvert \leq M, \lvert x^\prime -q^\prime \rvert > 2 \} \\
		C & \coloneqq E_0 \setminus S \\
		S & \coloneqq \{ (x^\prime,x_n) \text{ s.t. } \lvert x_n - q_n \rvert \leq M+R, \lvert x^\prime -q^\prime \rvert \leq 2\}
	\end{align*}

	Integration over the first part:
	\begin{align*}
		     & \int_A \frac{\chi_{E^c} -\chi_E}{\lvert y-q\rvert^{n+s}} \dd{y} \overset{A \subset E^c}{ \leq} \int_{\lvert y_n \rvert \geq R} \frac{1}{\lvert y \rvert^{n+s}} \dd{y} \leq c(n) \int_0^\infty \int_R^\infty \frac{r^{n-2}}{(r^2 +y_n^2)^{\frac{n+s}{2}}} \dd{y_n} \dd{r} \\
		\leq & \ c(n) \int_0^\infty \int_R^\infty \frac{1}{(r^2 +y_n^2)^{\frac{s+2}{2}}} \dd{y_n} \dd{r} \leq c(n) \int_0^\infty \int_R^\infty \frac{1}{(r+y_n)^{s+2}} \dd{y_n} \dd{r} \\
		=    & \ c(n,s) \int_0^\infty \frac{1}{(r+R)^{s+1}} \dd{r} = c(n,s) R^{-s}
	\end{align*}

	Integration over the second part:
	\begin{align*}
		     & \int_B \frac{\chi_{E^c} -\chi_E}{\lvert y-q\rvert^{n+s}} \dd{y} \overset{B \subset E^c}{ \leq} c(n) \int_0^M \int_2^\infty \frac{r^{n-2}}{(r^2 +y_n^2)^{\frac{n+s}{2}}} \dd{r} \dd{y_n} \\
		\leq & \ c(n) \int_0^M \int_2^\infty \frac{1}{( r+y_n)^{s+2}} \dd{r} \dd{y_n} = c(n,s) \int_0^M \frac{1}{(2+y_n)^{s+1}} \dd{r} \\
		=    & \ c(n,s)(2^{-s}-(2+M)^{-s}) \leq c(n,s) 2^{-s}
	\end{align*}

	Integration over the third part (here we need \( R > M \)):
	\begin{align*}
		\int_C \frac{\chi_{E^c} -\chi_E}{\lvert y-q\rvert^{n+s}} \dd{y} = - \int_C \frac{1}{\lvert y-q \rvert^{n+s}} \dd{y} \leq -c(n) \int_{B_M (\ldots)} \frac{1}{\lvert y\rvert^{n+s}} \dd{y} \leq -c(n,s) M^{-s}
	\end{align*}
	Idea: Move part of the stripe outside, restrict to ball with radius \( M \) and
	multiply with \( \frac{1}{2} \) since not whole ball may be in the set.\\

	Integration over the fourth part: \\
	We split \( S \) into four parts:
	\begin{enumerate}[label = \roman*)]
		\item \( S \cap B_{\Lambda M} (q) \cap B_{r_0}(z) \)
		\item \( S \cap B_{\Lambda M} (q) \cap B_{r_0}( \overline{z}) \)
		\item \( S \cap (B_{\Lambda M} (q)\setminus ( B_{r_0}(z) \cup B_{r_0}(z))) \)
		\item \( S \setminus B_{\Lambda M} (q) \)
	\end{enumerate}
	where \( \overline{z}\coloneqq z + 2(q-z) \) and \( \Lambda > 4 \) chosen big enough
	and \( M \) chosen small enough s.t.\ \( \Lambda M \leq 1 \). \\
	Again the first and second part are in sum smaller than zero. \\
	We estimate the third part:
	\begin{align*}
		     & \int_{S \cap (B_{\Lambda M} (q)\setminus ( B_{r_0}(z) \cup B_{r_0}(\overline{z})))} \frac{\chi_{E^c} -\chi_E}{\lvert y-q\rvert^{n+s}} \dd{y} \\
		\leq & \ \int_{P_{r_0, 1}} \frac{1}{\lvert y\rvert^{n+s}} \dd{y} + \int_{B_{\Lambda M}\setminus B_{r_0}} \frac{1}{\lvert y\rvert^{n+s}} \dd{y} \leq c(n,s) (r_0^{-s} - (\Lambda M)^{-s})
	\end{align*}
	We estimate the fourth part:
	\begin{align*}
		     & \int_{S \setminus B_{\Lambda M}(q)} \frac{\chi_{E^c} -\chi_E}{\lvert y-q\rvert^{n+s}} \dd{y} \\
		\leq & \ c(n) \int_{\Lambda M}^{R+3} \frac{1}{r^{s+1}} \dd{r} \leq c(n,s)((\Lambda M)^{-s} - (R+3)^{-s})
	\end{align*}
	Thus we estimate the domain \( S \) with
	\begin{align*}
		\int_S \frac{\chi_{E^c} -\chi_E}{\lvert y-q\rvert^{n+s}} \dd{y} \leq c(n,s)(r_0^{-s} - (R+3)^{-s}) \leq c(n,s) r_0^{-s}
	\end{align*}
	\par
	Thus in total we get:
	\begin{align*}
		\int_{\mathbb{R}^n} \frac{\chi_{E^c} -\chi_E}{\lvert y-q\rvert^{n+s}} \dd{y}
		 & \leq -c_0 M^{-s} +c_1 (R^{-s}+2^{-s} + r_0^{-s}) \\
		 & \leq -c_0 M^{-s}(1- \frac{c_1}{c_0} (R^{-s}M^s + 2^{-s}M^s + r_0^{-s}M^s))
		\intertext{Now choose \( r_0 = \frac{R}{2} \) and at most \( 2 \)}
		 & \leq -c_0 M^{-s}(1- \frac{c_1}{c_0} (R^{-s}M^s + 2^{-s}M^s + \left(\frac{2M}{R}\right)^s))
		\intertext{Choose \( \Lambda \) large and \( M \) small enoguh}
		 & \leq -c_2 M^{-s} < 0
	\end{align*}
\end{proof}

\section{Result 04}
\label{sec:005}

\begin{theorem}[Model 02: disconnected]
	\label{thm:004}
	Let the setting be as in theorem \cref{thm:003}, then there exists an \( M_0 \)
	s.t.\ for all \( M \geq M_0 \) the minimizer of the fractional perimeter is disconnected.
\end{theorem}
\begin{proof}
	Proof analogous to theorem 1.2 in 2016 dipierro-onoue-valdinoci. (I think) \\
\end{proof}

\section{Result 05}
\label{sec:006}


\end{document}
