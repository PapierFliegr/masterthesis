% -------------------------- tum template -------------------------- %
\makeatletter %redefine some labels from the TUM template
\provideName{\@tum@examiner@}{Supervisor}{Themensteller} % or `Themenstellerin`
\provideName{\@tum@supervisor@}{Advisor}{Betreuer} % or `Advisor` / `Betreuerin`
\makeatother

\title{TBD}
\subtitle{TBD}
\author{Mohamed Noah Abdel Wahab}
\degree{Master of Science}% or `Bachlor of Science`
\dateSubmitted{Date of submission}% preferably use some universally recognized date format
\examiner{Prof.\ Dr.\ Marco Cicalese}% `Themensteller`
\supervisor{Dr.\ Fumihiko Onoue}% `Betreuer`


% -------------------------- environments -------------------------- %
\mdfsetup{skipabove=1em, skipbelow=0em, innertopmargin=8pt, innerbottommargin=8pt}

\theoremstyle{definition}

\declaretheoremstyle[
    headfont=\bfseries\sffamily, bodyfont=\normalfont, mdframed={ nobreak } 
    ]{thmgreenbox}
\declaretheoremstyle[
    headfont=\bfseries\sffamily, bodyfont=\normalfont, mdframed={ nobreak } 
    ]{thmredbox}
\declaretheoremstyle[
    headfont=\bfseries\sffamily, bodyfont=\normalfont
    ]{thmbluebox}
\declaretheoremstyle[
    headfont=\bfseries\sffamily, bodyfont=\normalfont
    ]{thmblueline}
% \declaretheoremstyle[
%     headfont=\bfseries\sffamily, bodyfont=\normalfont, numbered=no, 
%     mdframed={ rightline=false, topline=false, bottomline=false, }, qed=\qedsymbol 
%     ]{thmproofbox}
\declaretheoremstyle[
    headfont=\bfseries\sffamily, bodyfont=\normalfont, numbered=no, mdframed={ nobreak,
    rightline=false, topline=false, bottomline=false } 
    ]{thmexplanationbox}

\declaretheorem[numberwithin=chapter, style=thmgreenbox, name=Definition]{definition}
% \declaretheorem[sibling=definition, style=thmredbox, name=Corollary]{corollary}
% \declaretheorem[sibling=definition, style=thmredbox, name=Proposition]{prop}
\declaretheorem[sibling=definition, style=thmredbox, name=Theorem]{theorem}
% \declaretheorem[sibling=definition, style=thmredbox, name=Lemma]{lemma}
\declaretheorem[sibling=definition, style=thmbluebox,  name=Example]{example}
% \declaretheorem[sibling=definition, style=thmbluebox,  name=Nonexample]{noneg}
% \declaretheorem[sibling=definition, style=thmblueline, name=Remark]{remark}

% \declaretheorem[numbered=no, style=thmexplanationbox, name=Proof]{explanation}
% \declaretheorem[style=thmbluebox,  numbered=no, name=Exercise]{ex}
\declaretheorem[style=thmblueline, numbered=no, name=Note]{note}

% \newtheorem*{uovt}{UOVT}
% \newtheorem*{notation}{Notation}
% \newtheorem*{previouslyseen}{As previously seen}
% \newtheorem*{problem}{Problem}
% \newtheorem*{observe}{Observe}
% \newtheorem*{property}{Property}
% \newtheorem*{intuition}{Intuition}


\declaretheoremstyle[
    headfont=\bfseries\sffamily\color{red!70!black}, bodyfont=\normalfont,
    mdframed={
        linewidth=2pt,
        rightline=false, topline=false, bottomline=false,
        linecolor=red, backgroundcolor=red!5,
    }
    ]{todo}
\declaretheorem[numbered=no, style=todo, name=TODO]{TODO}
\declaretheoremstyle[
    headfont=\bfseries\sffamily\color{green!70!black}, bodyfont=\normalfont,
    mdframed={
        linewidth=2pt,
        rightline=false, topline=false, bottomline=false,
        linecolor=green, backgroundcolor=green!5,
    }
    ]{check}
\declaretheorem[numbered=no, style=check, name=CHECK]{CHECK}