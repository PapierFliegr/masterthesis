% -------------------------- tum template -------------------------- %
\makeatletter %redefine some labels from the TUM template
\provideName{\@tum@examiner@}{Supervisor}{Themensteller} % or `Themenstellerin`
\provideName{\@tum@supervisor@}{Advisor}{Betreuer} % or `Advisor` / `Betreuerin`
\makeatother

\title{On the Connectedness of Nonlocal Minimal Surfaces in a Cylinder with (un)bounded Boundary Data}
\subtitle{}
\author{Mohamed Noah Abdel Wahab}
\degree{Master of Science}% or `Bachlor of Science`
\dateSubmitted{Date of submission}% preferably use some universally recognized date format
\examiner{Prof.\ Dr.\ Marco Cicalese}% `Themensteller`
\supervisor{Dr.\ Fumihiko Onoue}% `Betreuer`


\setlength\parindent{0pt}
% ---------------------------- ref setup --------------------------- %
\counterwithin*{equation}{chapter}
\newcommand\tagged{\stepcounter{equation}\tag{\theequation}}

\crefname{equation}{}{}


% -------------------------- environments -------------------------- %
\mdfsetup{skipabove=1em, skipbelow=0em, innertopmargin=4pt, innerbottommargin=4pt}

\declaretheoremstyle[
	headfont=\bfseries\sffamily, bodyfont=\normalfont, mdframed={ nobreak }
]{defbox}
\declaretheoremstyle[
	headfont=\bfseries\sffamily, bodyfont=\normalfont, mdframed={ nobreak }
]{thmbox}
\declaretheoremstyle[
	headfont=\bfseries\sffamily, bodyfont=\normalfont
]{remarkbox}
\declaretheoremstyle[
	headfont=\bfseries\sffamily, bodyfont=\normalfont
]{notebox}

\declaretheorem[numberwithin=chapter, style=defbox, name=Definition]{definition}
\declaretheorem[sibling=definition, style=thmbox, name=Theorem]{theorem}
\declaretheorem[sibling=definition, style=thmbox, name=Corollary]{corollary}
\declaretheorem[sibling=definition, style=thmbox, name=Proposition]{prop}
\declaretheorem[sibling=definition, style=thmbox, name=Lemma]{lemma}
\declaretheorem[sibling=definition, style=remarkbox,  name=Example]{example}
\declaretheorem[sibling=definition, style=remarkbox, name=Remark]{remark}
\declaretheorem[style=notebox, numbered=no, name=Note]{note}


% --------------------- structure environments --------------------- %
\declaretheoremstyle[
	headfont=\bfseries\sffamily\color{yellow!70!black}, bodyfont=\normalfont,
	mdframed={
			linewidth=2pt,
			rightline=false, topline=false, bottomline=false,
			linecolor=yellow, backgroundcolor=yellow!5,
		}
]{idea}
\declaretheorem[numbered=no, style=idea, name=IDEA]{IDEA}
\declaretheoremstyle[
	headfont=\bfseries\sffamily\color{red!70!black}, bodyfont=\normalfont,
	mdframed={
			linewidth=2pt,
			rightline=false, topline=false, bottomline=false,
			linecolor=red, backgroundcolor=red!5,
		}
]{todo}
\declaretheorem[numbered=no, style=todo, name=TODO]{TODO}
\declaretheoremstyle[
	headfont=\bfseries\sffamily\color{green!70!black}, bodyfont=\normalfont,
	mdframed={
			linewidth=2pt,
			rightline=false, topline=false, bottomline=false,
			linecolor=green, backgroundcolor=green!5,
		}
]{check}
\declaretheorem[numbered=no, style=check, name=CHECK]{CHECK}

% \usepackage{environ}
% \RenewEnviron{IDEA}{}{}
% \RenewEnviron{TODO}{}{}
% \RenewEnviron{CHECK}{}{}
