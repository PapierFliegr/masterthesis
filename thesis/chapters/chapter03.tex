%  Model 02
\chapter{Model 02}
\label{ch:model02}

For \( n \geq 2 \) consider the model as follows:
\begin{align}
	E_0    & \coloneqq \{(x^\prime,x_n) \in \mathbb{R}^{n-1} \times \mathbb{R} \text{ s.t. } M \lvert x_n \rvert \geq R+M \} \\
	\Omega & \coloneqq \{(x^\prime,x_n) \in \mathbb{R}^{n-1} \times \mathbb{R} \text{ s.t. } \lvert x^\prime \rvert \leq 1, \, \lvert x_n \rvert \leq M \}
\end{align}
for \( R > 0 \) and \( M > 0 \). The figure.. illustrates the setting.

%TODO: include figure

We state the following two results, which we will prove afterwards.

%TODO: Specify R
\begin{theorem}
	\label{thm:301}
	Let \( \Omega \) and \( E_0 \) as given above and for all \( R >.. \), then there
	exists \( M_0 \in (0,1) \) depending only on the dimension and \( s \), such that
	for any \( M \in (0, M_0) \), the minimizer is \( E_M = E_0 \cup \Omega \). For \( R
	\leq.. \), the cylinder \( A \coloneqq \{(x^\prime,x_n) \in \mathbb{R}^{n-1} \times
	\mathbb{R} \text{ s.t.~} \lvert x^\prime \rvert \leq.., \, \lvert x_n \rvert \leq M
	\} \) is in the minimizer, i.e.\ \( E_M \supset E_0 \cup A \).
\end{theorem}
\begin{note}
	Bound on \( R \) depends on the construction of the proof.
	%TODO: Elaborate 
\end{note}

\begin{theorem}
	\label{thm:302}
	For \( \Omega \) and \( E_0 \) as given above and for all \( R > 0 \), then there
	exists \( M_0 >.. \) depending only on the dimension and \( s \), such that
	for any \( M \geq M_0 \), the minimizer \( E_M \) is disconnected.
\end{theorem}

Again, similar proofs as in \cref{ch:model01}.\newline
Add some more discussion. 

\begin{proof}[Proof of \cref{thm:301}]
	We show that for every \( R > 0 \) at least the tube \( \{ \lvert x_n \rvert < r_0 \}
	\) is in the minimizer for some \( r_0 > 0 \). \\
	We do that analogously to theorem \cref{thm:-01} by contradiction. We assume that \(
	E_M \) is disconnected, thus we can slide a ball of radius \( r \) down and for all \(
	r_0 \in (0,1) \) there exists a \( t_0 > 0 \) s.t.\ \( \partial B_{r_0}(t_0 e_n) \cap
	\partial E_M \neq \emptyset \). If we can show that there exists a \( r_0 \) s.t.\
	this conntradicts then the tube is in the minimizer. It is enough to show that for one
	\( r_0 \) since if we can contradict this for one \( r_0 \) then for all smaller there
	is no touching as well. \\
	For that we split into four parts as seen in the figure:
	%TODO: Add figure
	We define
	\begin{align*}
		A & \coloneqq \{(x^\prime,x_n) \text{ s.t. } \lvert x_n \rvert \geq M+R \} \\
		B & \coloneqq \{(x^\prime,x_n) \text{ s.t. } \lvert x_n \rvert \leq M, \lvert x^\prime -q^\prime \rvert > 2 \} \\
		C & \coloneqq E_0 \setminus S \\
		S & \coloneqq \{ (x^\prime,x_n) \text{ s.t. } \lvert x_n - q_n \rvert \leq M+R, \lvert x^\prime -q^\prime \rvert \leq 2\}
	\end{align*}

	Integration over the first part:
	\begin{align*}
		     & \int_A \frac{\chi_{E^c} -\chi_E}{\lvert y-q\rvert^{n+s}} \dd{y} \overset{A \subset E^c}{ \leq} \int_{\lvert y_n \rvert \geq R} \frac{1}{\lvert y \rvert^{n+s}} \dd{y} \leq c(n) \int_0^\infty \int_R^\infty \frac{r^{n-2}}{(r^2 +y_n^2)^{\frac{n+s}{2}}} \dd{y_n} \dd{r} \\
		\leq & \ c(n) \int_0^\infty \int_R^\infty \frac{1}{(r^2 +y_n^2)^{\frac{s+2}{2}}} \dd{y_n} \dd{r} \leq c(n) \int_0^\infty \int_R^\infty \frac{1}{(r+y_n)^{s+2}} \dd{y_n} \dd{r} \\
		=    & \ c(n,s) \int_0^\infty \frac{1}{(r+R)^{s+1}} \dd{r} = c(n,s) R^{-s}
	\end{align*}

	Integration over the second part:
	\begin{align*}
		     & \int_B \frac{\chi_{E^c} -\chi_E}{\lvert y-q\rvert^{n+s}} \dd{y} \overset{B \subset E^c}{ \leq} c(n) \int_0^M \int_2^\infty \frac{r^{n-2}}{(r^2 +y_n^2)^{\frac{n+s}{2}}} \dd{r} \dd{y_n} \\
		\leq & \ c(n) \int_0^M \int_2^\infty \frac{1}{( r+y_n)^{s+2}} \dd{r} \dd{y_n} = c(n,s) \int_0^M \frac{1}{(2+y_n)^{s+1}} \dd{r} \\
		=    & \ c(n,s)(2^{-s}-(2+M)^{-s}) \leq c(n,s) 2^{-s}
	\end{align*}

	Integration over the third part (here we need \( R > M \)):
	\begin{align*}
		\int_C \frac{\chi_{E^c} -\chi_E}{\lvert y-q\rvert^{n+s}} \dd{y} = - \int_C \frac{1}{\lvert y-q \rvert^{n+s}} \dd{y} \leq -c(n) \int_{B_M (\ldots)} \frac{1}{\lvert y\rvert^{n+s}} \dd{y} \leq -c(n,s) M^{-s}
	\end{align*}
	Idea: Move part of the stripe outside, restrict to ball with radius \( M \) and
	multiply with \( \frac{1}{2} \) since not whole ball may be in the set.\\

	Integration over the fourth part: \\
	We split \( S \) into four parts:
	\begin{enumerate}[label = \roman*)]
		\item \( S \cap B_{\Lambda M} (q) \cap B_{r_0}(z) \)
		\item \( S \cap B_{\Lambda M} (q) \cap B_{r_0}( \overline{z}) \)
		\item \( S \cap (B_{\Lambda M} (q)\setminus ( B_{r_0}(z) \cup B_{r_0}(z))) \)
		\item \( S \setminus B_{\Lambda M} (q) \)
	\end{enumerate}
	where \( \overline{z}\coloneqq z + 2(q-z) \) and \( \Lambda > 4 \) chosen big enough
	and \( M \) chosen small enough s.t.\ \( \Lambda M \leq 1 \). \\
	Again the first and second part are in sum smaller than zero. \\
	We estimate the third part:
	\begin{align*}
		     & \int_{S \cap (B_{\Lambda M} (q)\setminus ( B_{r_0}(z) \cup B_{r_0}(\overline{z})))} \frac{\chi_{E^c} -\chi_E}{\lvert y-q\rvert^{n+s}} \dd{y} \\
		\leq & \ \int_{P_{r_0, 1}} \frac{1}{\lvert y\rvert^{n+s}} \dd{y} + \int_{B_{\Lambda M}\setminus B_{r_0}} \frac{1}{\lvert y\rvert^{n+s}} \dd{y} \leq c(n,s) (r_0^{-s} - (\Lambda M)^{-s})
	\end{align*}
	We estimate the fourth part:
	\begin{align*}
		     & \int_{S \setminus B_{\Lambda M}(q)} \frac{\chi_{E^c} -\chi_E}{\lvert y-q\rvert^{n+s}} \dd{y} \\
		\leq & \ c(n) \int_{\Lambda M}^{R+3} \frac{1}{r^{s+1}} \dd{r} \leq c(n,s)((\Lambda M)^{-s} - (R+3)^{-s})
	\end{align*}
	Thus we estimate the domain \( S \) with
	\begin{align*}
		\int_S \frac{\chi_{E^c} -\chi_E}{\lvert y-q\rvert^{n+s}} \dd{y} \leq c(n,s)(r_0^{-s} - (R+3)^{-s}) \leq c(n,s) r_0^{-s}
	\end{align*}
	\par
	Thus in total we get:
	\begin{align*}
		\int_{\mathbb{R}^n} \frac{\chi_{E^c} -\chi_E}{\lvert y-q\rvert^{n+s}} \dd{y}
		 & \leq -c_0 M^{-s} +c_1 (R^{-s}+2^{-s} + r_0^{-s}) \\
		 & \leq -c_0 M^{-s}(1- \frac{c_1}{c_0} (R^{-s}M^s + 2^{-s}M^s + r_0^{-s}M^s))
		\intertext{Now choose \( r_0 = \frac{R}{2} \) and at most \( 2 \)}
		 & \leq -c_0 M^{-s}(1- \frac{c_1}{c_0} (R^{-s}M^s + 2^{-s}M^s + \left(\frac{2M}{R}\right)^s))
		\intertext{Choose \( \Lambda \) large and \( M \) small enoguh}
		 & \leq -c_2 M^{-s} < 0
	\end{align*}
\end{proof}

Disscussion about connectedness in case of small \( R \) and refer to next chapter.
Behavior unique to nonlocal minimal surfaces.\newline
Talk about the contribution of the complement.

\begin{proof}[Proof of \cref{thm:302}]
	In theorems 1.2 in \cite{dipierro2020disconnectedness} the authors have shown that
	that \( \exists M_{0} > 1 \), such that ..  
	\begin{gather}
		E_{M} \subset F_{M} \quad E_{M} ^{c} \subset F_{M} ^{c}
	\end{gather}
\end{proof}

Discussion about extending the model to arbitrary models with symmetric external data.
Enough to consider discs of radius.. and heigth.. to have connectedness and even
stickiness at some point.\newline
New idea: If there is a minimizer \( E_M \), can it ever be non sticky to the boundary?\newline
Maybe able to give own interpretation of nonlocal minimal surfaces. Idea about Volume or
Gravity?

