\chapter{Introduction}
\label{ch:introduction}

\emph{Minimal surfaces} - During the 18th century mathematicians like \emph{Euler} and
\emph{Lagrange} concerned themselves with the problem of finding the set of smallest surface area
given some contour, often referred to as the \emph{Plateau Problem}. The solution of such a problem is
called \emph{Minimal Surface} and the work of Euler and Lagrange laid the foundation for the study
of those. While not solving the Plateau Problem themselves, they found necessary conditions on these
surfaces, namely the \emph{Euler-Lagrange equations}~\cite{Maggi_2012}. Nowadays, we have multiple
tools to define and characterize minimal surfaces, among those the concept known as
\emph{Perimeter}. The modern understanding of the perimeter is due to the work of \emph{Caccioppoli}
and \emph{De Giorgi} in the 20th century~\cite{Maggi_2012}. Roughly speaking, the perimeter of a set
can be understood as the area of the boundary of the set, more on that in \Cref{sec:001}. With the
concept of the perimeter, we can define a minimal surface as the set which has the smallest
perimeter given some external data. In modern times, minimal surfaces have found many applications,
from understanding physical phenomena like soap films and black holes to informing the design of
optimal structures in engineering and architecture.\\

In this thesis, we want to explore a rather recent concept of minimal surfaces, namely
\emph{Nonlocal Minimal Surfaces}, which were first introduced by \emph{Cafarelli},
\emph{Roquejoffre}, and \emph{Savin} in 2010~\cite{caffarelli2009nonlocal}. For that purpose, we will first give a short
introduction to the theory of minimal surfaces in the context of this work.

\section{Classical Minimal Surfaces}
\label{sec:001}

The study of minimal surfaces concerns itself with finding the set of least surface area or
\emph{perimeter} under certain constrains. But before we can formulate the usual problem, we have to
define some tools~\cite{Cozzi2017}.

\begin{definition}
	Let \( E \subset \mathbb{R}^n \) be a set with smooth boundary, then the surface area or
	\emph{perimeter}
	of \( E \) is given by
	\begin{gather*}
		\Per{}{E}{} \coloneqq \sup \left\{\int_{\partial E} \varphi \cdot \nu_E \dd{\mathcal{H}^{n - 1}} \mid \varphi \in C_c^1 (\mathbb{R}^n, \mathbb{R}^n), \lvert \varphi \rvert \leq 1 \right\},
	\end{gather*}
	where \( \nu_E \) is the outer normal to \( E \).
\end{definition}

The \emph{perimeter} of a set can be understood as the surface area of the set's boundary. An easy
example would be the perimeter of a circle, which is just the circumference of the circle. To extend
this definition to general measurable sets, we can use the divergence theorem and rewrite the
integration over the boundary as an integration over the set itself. This removes the need for a
smooth boundary and allows us to define the surface area for general sets.

\begin{definition}
	Let \( E \subset \mathbb{R}^n \), then the perimeter of \( E \) is given by
	\begin{gather*}
		\Per{}{E}{} \coloneqq \sup \left\{\int_E \div \varphi \mid \varphi \in C_c^1 (\mathbb{R}^n, \mathbb{R}^n), \lvert \varphi \rvert \leq 1 \right\}.
	\end{gather*}
\end{definition}

Even now, without the explicit integration over the boundary of the set, we can understand the
perimeter in a sense as the surface area of the boundary. In fact, De Giorgi and Federer have shown
that the perimeter of a set coincides with the \( (n\!-\!1) \)-dimensional Hausdorff measure of a
suitable subset of the topological boundary of the set~\cite{Fleming_2020}. This supports the idea
that the perimeter of a set can be understood as the surface area of the set's boundary. \\

Usually we are interested in the minimization of the perimeter of a set relative to some external
data. Thus, we need a tool which allows us to localize the area of interest.

\begin{definition}
	Let \( E \subset \mathbb{R}^n \) and take \( \Omega \subset \mathbb{R}^n \) bounded,
	then the perimeter of \( E \) relative to \( \Omega \) is given by
	\begin{gather*}
		\Per{}{E}{\Omega} \coloneqq \sup \left\{\int_E \div \varphi \mid \varphi \in C_c^1 (\Omega, \mathbb{R}^n), \lvert \varphi \rvert \leq 1 \right\}.
	\end{gather*}
\end{definition}

With the necessary tools defined we can formulate the usual problem.
\begin{definition}[Minimal Surface Problem]
	\label{def:minimal_surface_problem}
	Let \( \Omega \subset \mathbb{R}^n \) bounded and \( E_0 \subset \mathbb{R}^n \), then we want to
	find \( E \subset \mathbb{R}^n \) such that \( E \) minimizes the perimeter of \( E_0 \) relative
	to \( \Omega \), i.e.
	\begin{gather*}
		\Per{}{E}{\Omega} = \min \left\{\Per{}{F}{\Omega} \mid F \setminus \Omega = E_0 \setminus \Omega \right\}.
	\end{gather*}
	This set \( E \) is then called a \emph{minimal surface}.
\end{definition}

\begin{note}
	We use the term \enquote{minimal surface} in the sense that we minimize over the area of surfaces,
	whereas in the context of differential geometry one sometimes uses the term \enquote{minimal
		surfaces} for critical points of the area functional, see~\cite{meeks2012survey}.
\end{note}

For existence and other interesting properties of the classical perimeter and minimal surfaces not
needed here in this work, we refer
to~\cite{Cozzi2017},~\cite{Evans2015},~\cite{Giusti1984},~\cite{Grinfeld2013} and~\cite{Maggi_2012}.

\section{Nonlocal Minimal Surfaces}
\label{sec:002}

A standard example of a minimal surface are soap films~\cite{Serra2023}. Imagine you have a some
wire, dip it in soap and observe the soap film that forms. This film is a minimal surface, i.e.\ it
has the smallest surface area for the given contour (the wire). Observing the soap film with the
naked eye, it seems like the soap film is a \( 2 \)-dimensional object, see \Cref{fig:002}, thus to
minimize the area we only need to consider each point and its immediate neighborhood in the film,
i.e.\ we only need to minimize locally\footnote{Refers to the proximity of a point and not the area
	of interest}.
\begin{figure}[ht]
	\centering
	\def\svgscale{1}
	\import{figures/soap_film}{soap_film.pdf_tex}
	\caption{Soap film formed by a wire. Soap molecules are represented by the orange dots.}
	\label{fig:002}
\end{figure}
If we consider some set \( E \subset \mathbb{R}^n \) with smooth
boundary, then to obtain its perimeter we have to compute
\begin{gather*}
	\int_{\partial E} \varphi \cdot \nu_E.
\end{gather*}
Notice that this is a local quantity, i.e.\ it only depends on the boundary of \( E \). Now let us
zoom in on the soap film. Zooming in on the soap film at molecular scale, we see that the soap film
is a \( 3 \)-dimensional object, see \Cref{fig:002}, and the classical theory of minimal surfaces
does not suffice anymore. Notice that we can no longer just minimize over the boundary of the set,
but also need to minimize the volume of the set itself, thus we need to incorporate long-range
correlation into the definition of the perimeter and minimal surfaces. This is where the concept of
a \emph{Fractional Perimeter} comes into play as introduced by \emph{Cafarelli}, \emph{Roquejoffre}
and \emph{Savin} in 2010~\cite{caffarelli2009nonlocal}.


\begin{definition}[Fractional Perimeter]
	\label{def:fractional_perimeter}
	Let \( E \subset \mathbb{R}^n \) be a Borel set, \( s \in (0, 1) \), then the \( s \)-perimeter
	or fractional perimeter of \( E \) is given by
	\begin{gather*}
		\Per{s}{E}{} \coloneqq \int_E \int_{E^c} \frac{1}{\lvert x - y \rvert^{n + s}} \dd{y} \dd{x}.
	\end{gather*}
\end{definition}

\begin{corollary}
	\label{cor:001}
	The fractional perimeter is translation invariant and positive homogeneous of grade \( n-s \).
\end{corollary}
\begin{proof}
	Let \( E \subset \mathbb{R}^n \) be a Borel set, \( s \in (0, 1) \). Take any \( h \in
	\mathbb{R}^n \), then
	\begin{align*}
		\Per{s}{E + h}{}
		= \int_{E + h} \int_{(E + h)^c} \frac{1}{{\lvert x - y \rvert}^{n + s}} \dd{y} \dd{x}
		= \int_E \int_{E^c} \frac{1}{{\lvert x - y \rvert}^{n + s}} \dd{y} \dd{x} = \Per{s}{E}{}.
	\end{align*}
	Now take any \( \lambda > 0 \), then
	\begin{align*}
		\Per{s}{\lambda E}{}
		= \int_{\lambda E} \int_{(\lambda E)^c} \frac{1}{{\lvert x - y \rvert}^{n + s}} \dd{y} \dd{x}
		= \int_E \int_{E^c} \frac{\lambda^{2n}}{{\lvert \lambda x - \lambda y \rvert}^{n + s}} \dd{y} \dd{x}
		= \lambda^{n-s} \Per{s}{E}{}.
	\end{align*}
\end{proof}

Intuitively, we can understand the parameter \( s \) as the grade of nonlocality. Indeed, for
smaller \( s \) the contribution of points further away from each other have a bigger impact on the
fractional perimeter.

\begin{note}
	Here we consider the definition of the fractional perimeter given by the authors in~\cite{Caffarelli2011}. 
  Other papers like~\cite{mazon2019nonlocal} define a more
	general definition of the fractional perimeter with some nonlocal kernel \( K: \mathbb{R}^n \to
	[0, \infty] \), \( K \not\equiv \infty \)
	\begin{gather*}
		\Per{K}{E}{} \coloneqq \int_E \int_{E^c} K(x - y) \dd{y} \dd{x}.
	\end{gather*}
	Our definition is a special case of this more general definition with the kernel \( K(x) =
	\frac{1}{\lvert x \rvert^{n + s}} \).
\end{note}

\begin{note}
	In some literature like~\cite{Cozzi2017}, the fractional perimeter is sometimes defined with a
	factor \( 2 \) in front of the integral. This is just a convention to relate the fractional
	perimeter to the Gagliardo seminorm
	\begin{gather*}
		\lVert f \rVert_{W^{s, 1}(\mathbb{R}^n)} = \int_{\mathbb{R}^n} \int_{\mathbb{R}^n} \frac{\lvert f(x) - f(y) \rvert}{\lvert x - y \rvert^{n + s}} \dd{y} \dd{x}
	\end{gather*}
	to the fractional perimeter. Notice that
	\begin{gather*}
		\Per{s}{E}{} = \int_E \int_{E^c} \frac{1}{\lvert x - y \rvert^{n + s}} \dd{y} \dd{x} = \frac{1}{2} \int_{\mathbb{R}^n} \int_{\mathbb{R}^n} \frac{\lvert \chi_E (x) - \chi_E (y) \rvert}{\lvert x - y \rvert^{n + s}} \dd{y} \dd{x} = \frac{1}{2} \lVert \chi_A \rVert_{W^{s, 1}(\mathbb{R}^n)},
	\end{gather*}
	i.e.\ the fractional perimeter is the seminorm of the indicator function of \( E \) up to a
	multiplicative constant.
\end{note}

The fractional perimeter can be seen as a generalization of the classical perimeter. This is among
other reasons due to the fact that for a set of finite \emph{classical} perimeter, say \( E \), we
have that \( (1-s) \Per{s}{E}{} \to c \Per{}{E}{} \) as \( s \nearrow 1 \) for some dimensional
constant \( c > 0 \)~\cite{Caffarelli2011}. Another argument for the relation between the fractional perimeter
and the classical perimeter is that the fractional perimeter can be seen as the \( W^{s, 1} \)
Gagliardo seminorm, as mentioned before. Thus, for \( s \nearrow 1 \) we can see the natural
connection to the classical perimeter, which is in some sense the \( W^{1, 1} \) seminorm of \(
\chi_E \)~\cite{Evans2015}. \\

Let us now consider an example where the classical perimeter does not suffice and the fractional
perimeter gives us better results. The example was discussed in~\cite{cinti2016quantitative} and~\cite{Serra2023}.

\begin{example}
	Consider a grid of square pixels of length \( \rho \) and a unit square rotated by \( 45 \) degrees, see \Cref{fig:001}.

	\begin{figure}[ht]
		\centering
		\def\svgscale{1}
		\import{figures/rotated_square}{rotated_square.pdf_tex}
		\caption{Unit square displayed by a grid of square pixels of length \( \rho \). Orange the actual
			boundary of the square, Blue the boundary of the pixelated square.}
		\label{fig:001}
	\end{figure}

	While the perimeter of the actual square is \( 4 \), the classical perimeter of the unit square
	composed of pixels is \( 4 \sqrt{2} \) independent of the size of the pixels \( \rho \). Indeed,
	we have that
	\begin{gather*}
		\Per{}{\text{Pixelated Square}}{} = 2 \rho \cdot (\text{number of pixels on the boundary}) = 8 \rho \cdot
		\frac{1}{\sqrt{2} \rho} = 4 \sqrt{2}.
	\end{gather*}
	We assume that the number of pixels on the boundary is a natural number for simplicity. We see,
	that the classical perimeter does not suffice to capture the actual perimeter of the square
	accurately, even for small pixel sizes.

	Now let us consider the fractional perimeter of the pixelated square. Let \( Q \) be the unit
	square, \( Q_\rho \) the pixelated square and \( N \in \mathbb{N} \) be the number of pixels on the
	boundary of \( Q \). Notice that \( Q_\rho \) is the union of \( Q \) and \( N \) disjoint isosceles
	triangles \( T_i \). We then have (omit the argument for simplicity)
	\begin{align*}
		\Per{s}{Q_\rho}{}
		 & = \int_{Q_\rho} \int_{Q_\rho^c}
		= \int_Q \int_{Q_\rho^c} + \int_{\cup_{i = 1}^N T_i} \int_{Q_\rho^c}
		= \int_Q \int_{Q^c} - \int_Q \int_{\cup_{i = 1}^N T_i} + \int_{\cup_{i = 1}^N T_i} \int_{Q_\rho^c} \\
		 & = \Per{s}{Q}{} + \Per{s}{\cup_{i = 1}^N T_i}{} - 2 \int_Q \int_{\cup_{i = 1}^N T_i}. \tagged\label{eq:001}
	\end{align*}

	We can bound \( \Per{s}{\cup_{i = 1}^N T_i}{} \) from above by
	\begin{align*}
		\Per{s}{\cup_{i = 1}^N T_i}{} \leq \sum_{i = 1}^N \Per{s}{T_i}{} \leq \sum_{i = 1}^N \rho^{2-s} \Per{s}{T}{} = N \rho^{2-s} \Per{s}{T}{} \leq c \rho^{1-s},
	\end{align*}
	where we scaled \( T_i \) to the Triangle \( T \) with side length \( 1 \) and \( c \) is a
	constant. With the dominated convergence theorem we can show that the last term in~\eqref{eq:001} goes
	to for \( \rho \to 0 \). Indeed, we have \( \chi_{\cup_{i = 1}^N T_i {}} \to 0 \) a.e.\ for \(
	\rho \to 0 \) and \( \int_Q \int_{\cup_{i = 1}^N T_i} \leq \Per{s}{Q}{} < \infty \) for all \( \rho \)~\cite{onoue2022}. \\
	Thus, the discrepancy between the fractional perimeter of the pixelated square and the actual
	square, that is \( \lvert \Per{s}{Q}{} - \Per{s}{Q_\rho}{} \rvert \), behaves like \(
	\rho^{1-s} + a(\rho) \) for some function \( a \) such that \( a(\rho) \to 0 \) as \( \rho \to 0
	\). This shows that the fractional perimeter can capture the actual perimeter of the square more
	accurately and serves as a more robust framework than the classical perimeter in some cases.
\end{example}

\vspace{1em}
Just as in the classical case, we can define a relative fractional perimeter by removing the
integration over some constant part. We split up the domain of integration (omit the argument for simplicity)
\begin{gather*}
	\int_E \int_{E^c}
	= \int_{E \cap \Omega} \int_{E^c} + \int_{E \setminus \Omega} \int_{E^c}
	= \int_{E \cap \Omega} \int_{E^c} + \int_{E \setminus \Omega} \int_{\Omega \setminus E} + \int_{E \setminus \Omega} \int_{E^c \setminus \Omega}.
\end{gather*}
While minimizing \( E \) relative to \( \Omega \) we can ignore the last term. Since we are only
interested in the part of the minimizer \( E \) in \( \Omega \), the last term is constant and
thus does not affect the minimization.
\begin{definition}
	Let \( A, B \subset \mathbb{R}^n \) be Borel sets, \( s \in (0, 1) \), then the interaction of \(
	A \) and \( B \) is given by
	\begin{gather*}
		\L (A, B) \coloneqq \int_A \int_{B^c} \frac{1}{\lvert x - y \rvert^{n + s}} \dd{y} \dd{x}.
	\end{gather*}
\end{definition}
\begin{note}
	In particular, we have that \( \Per{s}{E}{} = \L(E, E^c) \).
\end{note}
\begin{definition}[Relative Fractional Perimeter]
	\label{def:relative_fractional_perimeter}
	Let \( E \subset \mathbb{R}^n \) be a Borel set, \( \Omega \subset \mathbb{R}^n \) bounded and \(
	s \in (0, 1) \), then the \( s \)-perimeter of \( E \) relative to \( \Omega \) is given by
	\begin{gather*}
		\Per{s}{E}{\Omega} \coloneqq \L (E \cap \Omega, E^c) + \L(E \setminus \Omega, \Omega \setminus E).
	\end{gather*}
\end{definition}

With these tools we can now define the nonlocal minimal surface problem.
\begin{definition}[Nonlocal Minimal Surface Problem]
	\label{def:nonlocal_minimal_surface_problem}
	Let \( \Omega \subset \mathbb{R}^n \) bounded and \( E_0 \subset \mathbb{R}^n \), then we want to
	find \( E \subset \mathbb{R}^n \) such that \( E \) minimizes the \( s \)-perimeter of \( E_0 \)
	relative to \( \Omega \), i.e.
	\begin{gather*}
		\Per{s}{E}{\Omega} = \min \left\{\Per{s}{F}{\Omega} \mid F \setminus \Omega = E_0 \setminus \Omega \right\}.
	\end{gather*}
\end{definition}

Over the last few years these nonlocal minimal surfaces have been an area of great interest. Various
properties have been studied and numerous results have been obtained. Some of the more notable ones are
a monotonicity formula, see~\cite{caffarelli2009nonlocal}, and enhanced regularity properties
compared to classical minimal surfaces, see~\cite{caselli2024yaus}
and~\cite{millot2016asymptotics}. \\

An important tool in the study of minimal surfaces are the \emph{Euler-Lagrange equations}. These
equations give us necessary conditions for a set to be a minimal surface. In the case of classical
minimal surfaces these equations with the right argument are even sufficient~\cite{Maggi_2012}. The
authors in~\cite{caffarelli2009nonlocal} have shown that nonlocal minimal surfaces also have to satisfy a
nonlocal version of the Euler-Lagrange equations in the viscosity sense. \\
Let \( E \subset \mathbb{R}^n \) be a nonlocal minimal surface relative to some set \( \Omega \). If
\( E \cap \Omega \) has an interior tangent ball at some point \( x \in \partial E \cap \Omega \),
then
\begin{gather*}
	\int_{\mathbb{R}^n} \frac{\chi_{E^c} (y) - \chi_E (y)}{\lvert y-x \rvert^{n + s}} \dd{y} \geq 0.
\end{gather*}
Notice that this is a nonlocal version of the classical mean curvature equation given by \( H_E =
0 \) on \( \partial E \), where \( H_E \) is the mean curvature on \( \partial E \), in particular a
local entity. This gives us another reason as to why we refer to \emph{nonlocal} minimal surfaces.
For further details on the Euler-Lagrange equations for minimal surfaces we refer
to~\cite{caffarelli2009nonlocal} and~\cite{Maggi_2012}. \\

An interesting property unique to nonlocal minimal surfaces is the so-called \emph{stickiness
	property}. With \emph{stickiness}, we refer to the case that the boundary of the minimizer and
prescribed set intersect on a measurable set, i.e.\ \( \mathcal{H}^{n-1}(\partial E \cap \partial
\Omega) \neq 0 \). While in the classical case minimal surfaces cannot stick to the boundary of
the prescribed set, nonlocal minimal surfaces can stick. That is, let \( E \) be a classical minimal
surface relative to some prescribed set \( \Omega \), then \( \partial E \) and \( \partial \Omega
\) are transverse, see~\cite{Duzaar2000} and~\cite{hardt1979}.
In~\cite{dipierro2020disconnectedness} the authors have given an explicit example of a nonlocal
minimal surface which sticks to the boundary of the prescribed set. In this thesis we will
generalize this example and show that for those models the stickiness property persists. \\

In this thesis, we want to explore more on these surfaces and their properties, in particular the
connectedness of minimizers of certain models and the stickiness property in the same context. In
\Cref{ch:models} we will consider a generalization of the model given
in~\cite{dipierro2020disconnectedness}. In \Cref{ch:disconnected_minimizer} we will discuss a
natural question emerging while analyzing the models in \Cref{ch:models}, that is the existence of a
nontrivial minimizer in the case that the external data and the prescribed set have nonzero
distance. We will provide an example where such a minimizer exists. This behavior is unique to
nonlocal minimal surfaces.
