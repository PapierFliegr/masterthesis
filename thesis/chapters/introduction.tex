% Statements
\chapter{Introduction}
\label{ch:introduction}

Idea: Start with short historical background\\
18th century: Lagrange, Euler\\
20th Century: DeGiorgi Perimeter and localized entity\\
2009 Cafarelli, Roquejoffre, Savin: Nonlocal minimal surfaces\\
Perimeter and nonlocal perimeter as the (semi)norm of an indicator function\\
Define the usual problem considered\\
Better regularity than classical minimal surfaces\\

Chapter 02\\
Both models and further discussion

Chapter 03\\
Fully disconnected minimizer


Use the introduction in~\cite{dipierro2012asymptotics} as inspiration.

\begin{TODO}
	Add sources \\
	What does \enquote{locally} mean here? And what do we minimize? The surface or the
	area the encompassed?
\end{TODO}


\emph{Minimal surfaces}, characterized by locally minimizing their surface area, have
captivated mathematicians for centuries. Dating back to the 18th century, mathematicians
like \emph{Euler} and \emph{Lagrange} laid the foundation for the field. In an effort to
describe these surfaces mathematically, they formulated the \emph{Euler-Lagrange
	equations} in the late 18th century. These equations provide a powerful framework for
identifying and characterizing minimal surfaces. Since the 19th century, many
mathematicians contributed to the study of minimal surfaces, uncovering profound insights.
Since then minimal surfaces found many applications in various fields beyond pure
mathematics. From understanding physical phenomena like soap films and black holes to
informing the design of optimal structures in engineering and architecture, the
versatility of minimal surfaces continues to inspire exploration.\newline

In this thesis, we want to explore a rather recent concept of minimal surfaces, namely
\emph{nonlocal minimal surfaces}, which were first introduced by \emph{Cafarelli},
\emph{Roquejoffre}, and \emph{Savin} in 2009. For that purpose, we will first give a short
introduction to the theory of minimal surfaces in the context of this work.

\section{Classical Minimal Surfaces}
\label{sec:001}

\begin{CHECK}
	Is this introduction enough and complete/correct?
\end{CHECK}
The study of minimal surfaces concerns itself with finding the set with least surface area
under certain constrains. But before we can formulate the usual problem, we have to define
some tools.

\begin{TODO}
	Add definiton for smooth boundary, give an example (circle) \\
	Extend to general sets, give an example (cube)
\end{TODO}

\begin{CHECK}
	Do I need to cite this definition from~\cite{Cozzi2017}? Give a justification?
\end{CHECK}
\begin{definition}
	Let \( A \subset \mathbb{R}^n \) with smooth boundary, then the surface area or
	\emph{perimeter} of \( A \) is given by
	\begin{gather}
		\Per{}{A}{} \coloneqq \sup \left\{ \int_{\partial A} \varphi \cdot \nu_A \mid \varphi \in C_c^1 (\mathbb{R}^n, \mathbb{R}^n), \lvert \varphi \rvert \leq 1 \right\},
	\end{gather}
	where \( \nu_A \) is the outer normal to \( A \).
\end{definition}

\begin{CHECK}
	Do I need an example? Seems trivial
\end{CHECK}
\begin{example}
	Consider \( A = B_1 \subset \mathbb{R}^2 \), then
	\begin{gather}
		\left\lvert \int_{\partial A} \varphi \cdot \nu_A \dd{x} \right\rvert \leq \int_{\partial A} \lvert \varphi \rvert \lvert \nu_A \rvert \dd{x} = \int_{\partial A} \dd{x} = 2 \pi.
	\end{gather}
	Now take \( \varphi = \eta_3 \nu_A \) with \( \eta_3 \) a cutoff function such that \(
	\restr{\eta_3}{B_2} \equiv 1 \), \( \restr{\eta_3}{B_3^c} \equiv 0 \) and \( \eta_3
	\nu_A \in C_c^1 \) since \( \partial A \) smooth, then
	\begin{gather}
		\int_{\partial A} \varphi \cdot \nu_A \dd{x} = \int_{\partial A} \eta_3 \nu_A \cdot \nu_A \dd{x} = \int_{\partial A} \dd{x} = 2 \pi.
	\end{gather}
	Thus we have \( \area(\partial A) = 2 \pi \), as well known.
\end{example}

To extend this definition to general measurable sets, we can use the divergence theorem
and rewrite the integration over the boundary as an integration over the set itself. This
is removes the need for a smooth boundary and allows us to define the surface area for
general sets.
\begin{definition}
	Let \( A \subset \mathbb{R}^n \) be a Borel set, then the perimeter of \( A \) is
	given by
	\begin{gather}
		\Per{}{A}{} \coloneqq \sup \left\{ \int_A \div \varphi \mid \varphi \in C_c^1 (\mathbb{R}^n, \mathbb{R}^n), \lvert \varphi \rvert \leq 1 \right\}.
	\end{gather}
\end{definition}

\begin{CHECK}
	Do I need an example?
\end{CHECK}
\begin{example}
	Cube example
\end{example}

In the minimization problem, we want to find some set \( E \) which minimizes the surface
of some external data \( E_0 \). Since the surface area may be infinite, if \( E_0 \) is
bounded, we can \enquote{localize}\footnote{Here \enquote{local} refers to the area in
	which we minimize} the problem by just considering the area of \( \partial E \) relative
to some bounded set \( \Omega \).
\begin{definition}
	Let \( A \subset \mathbb{R}^n \) be a Borel set and \( \Omega \subset \mathbb{R}^n \)
	bounded, then the perimeter of \( A \) relative to \( \Omega \) is given by
	\begin{gather}
		\Per{}{A}{\Omega} \coloneqq \sup \left\{ \int_A \div \varphi \mid \varphi \in C_c^1 (\Omega, \mathbb{R}^n), \lvert \varphi \rvert \leq 1 \right\}.
	\end{gather}
\end{definition}

Now we can formulate the usual problem.
\begin{definition}[Minimal Surface Problem]
	\label{def:minimal_surface_problem}
	Let \( \Omega \subset \mathbb{R}^n \) bounded and \( E_0 \subset \mathbb{R}^n \), then
	we want to find \( E \subset \mathbb{R}^n \) such that \( E \) minimizes the perimeter
	of \( E_0 \) relative to \( \Omega \), i.e.
	\begin{gather}
		\Per{}{E}{\Omega} = \min \left\{ \Per{}{A}{\Omega} \mid A \setminus \Omega = E_0 \setminus \Omega \right\}.
	\end{gather}
\end{definition}
\begin{TODO}
	Complete note\\
	Case that \( E_0 \cap \Omega \neq \emptyset \)..\\
	Give sources, that minimizer exists, thus minimal surfaces exist\\
	Note that in classical theory often one just has a contour over which one minimizes
\end{TODO}
\begin{note}
	Usually \( E_0 \) is chosen such that \( E_0 \cap \Omega = \emptyset \), then we
	minimize over the set \( E \) such that \( E \setminus \Omega = E_0 \). If \( E_0 \cap
	\Omega \neq \emptyset \), then we can minimize over..
\end{note}



\section{Nonlocal Minimal Surfaces}
\label{sec:002}

\begin{TODO}
	Rewrite the text\\
	Is the example fitting?\\
	Emphasize that we are no longer just minimizing boundary but the set as well
\end{TODO}

Let us for now consider some set \( A \subset \mathbb{R}^n \) with smooth boundary, then
to get its perimeter we have to take the supremum of
\begin{gather}
	\int_{\partial A} \varphi \cdot \nu_A.
\end{gather}
This is a local quantity, i.e. it only depends on the boundary of \( A \). Thus if we want
to minimize the perimeter of some set \( E \) with external data \( E_0 \), we are only
interested in the behavior of the boundary of \( E_0 \) close to \( \Omega \) and not
interested in the contribution or the size of the external data. In many cases, this is
enough to describe the behavior of the minimizer, but in some cases, this is not enough
anymore. Take a soap bubble as an example, a standard example for a classical minimal
surfaces. In our normal scaling, we can see the soap bubble as a \( 2 \)-dimensional
object. But if we go to the molecular level, we see that the soap bubble is a \( 3 \)
-dimensional object. Thus we need to incorporate long-range correlation into our
definition of perimeter and minimal surfaces. \emph{Cafarelli}, \emph{Roquejoffre}, and
\emph{Savin} did exactly that in 2009, when they introduced the concept of \emph{nonlocal
	minimal surfaces} and \emph{fractional perimeter} in~\cite{caffarelli2009nonlocal}.

\begin{TODO}
	What is the effect of \( s \)? \\
	Which definition is standard? Add note about other definitions\\
\end{TODO}
\begin{definition}[Fractional Perimeter]
	\label{def:fractional_perimeter}
	Let \( A \subset \mathbb{R}^n \) be a Borel set, \( s \in (0, 1) \), then the \( s
	\)-perimeter of \( A \) to is given by
	\begin{gather}
		\Per{s}{A}{} \coloneqq \int_A \int_{A^c} \frac{1}{\lvert x - y \rvert^{n + s}} \dd{y} \dd{x}.
	\end{gather}
\end{definition}

Just as in the classical case, we can define a relative fractional perimeter by removing
the integration over the constant part..
\begin{align}
	 & \int_A \int_{A^c} \frac{1}{\lvert x - y \rvert^{n + s}} \dd{y} \dd{x} \\
	 & = \int_{A \cap \Omega} \int_{A^c} \frac{1}{\lvert x - y \rvert^{n + s}} \dd{y} \dd{x} + \int_{A \setminus \Omega} \int_{A^c} \frac{1}{\lvert x - y \rvert^{n + s}} \dd{y} \dd{x} \\
	 & = \int_{A \cap \Omega} \int_{A^c} \frac{1}{\lvert x - y \rvert^{n + s}} \dd{y} \dd{x} + \int_{A \setminus \Omega} \int_{\Omega \setminus A} \frac{1}{\lvert x - y \rvert^{n + s}} \dd{y} \dd{x} + \int_{A \setminus \Omega} \int_{A^c \setminus \Omega} \frac{1}{\lvert x - y \rvert^{n + s}} \dd{y} \dd{x}
\end{align}
While minimizing \( A \) relative to \( \Omega \) we can ignore the last term as it is
constant and thus does not affect the minimization.
\begin{definition}
	Let \( A, B \subset \mathbb{R}^n \) be Borel sets, \( s \in (0, 1) \), then the
	interaction of \( A \) and \( B \) is given by
	\begin{gather}
		\L (A, B) \coloneqq \int_A \int_{B^c} \frac{1}{\lvert x - y \rvert^{n + s}} \dd{y} \dd{x}.
	\end{gather}
\end{definition}
\begin{definition}[Relative Fractional Perimeter]
	\label{def:relative_fractional_perimeter}
	Let \( A \subset \mathbb{R}^n \) be a Borel set, \( \Omega \subset \mathbb{R}^n \)
	bounded and \( s \in (0, 1) \), then the \( s \)-perimeter of \( A \) relative to \( \Omega \) is given by
	\begin{gather}
		\Per{s}{A}{\Omega} \coloneqq \L (A \cap \Omega, A^c)+\L(A \setminus \Omega, \Omega \setminus A).
	\end{gather}
\end{definition}

With these tools we can now define the nonlocal minimal surface problem.
\begin{definition}[Nonlocal Minimal Surface Problem]
	\label{def:nonlocal_minimal_surface_problem}
	Let \( \Omega \subset \mathbb{R}^n \) bounded and \( E_0 \subset \mathbb{R}^n \), then
	we want to find \( E \subset \mathbb{R}^n \) such that \( E \) minimizes the \( s
	\)-perimeter of \( E_0 \) relative to \( \Omega \), i.e.
	\begin{gather}
		\Per{s}{E}{\Omega} = \min \left\{ \Per{s}{A}{\Omega} \mid A \setminus \Omega = E_0 \setminus \Omega \right\}.
	\end{gather}
\end{definition}

\begin{TODO}
	give an example where classical theory doesn't suffice (cube rotated by 45 degree)
\end{TODO}

\begin{TODO}
	Add note about advantages/properties (e.g. Euler-Lagrange Viscos) of nonlocal minimal
	surfaces like better regularity properties and..\\
	Add some sentences about stickiness property and that we are looking at a model
	precisely about that property.
\end{TODO}

\begin{TODO}
	Quick summary of \Cref{ch:models}
\end{TODO}

\begin{TODO}
	Quick summary of \Cref{ch:disconnected_minimizer}
\end{TODO}