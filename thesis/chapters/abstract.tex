\chapter{Abstract}
\label{ch:abstract}

\section*{Abstract}

Nonlocal minimal surfaces confined within a cylinder exhibit unique behaviors dependent on external
data. This thesis delves into these surfaces, which in contrast to classical minimal surfaces
incorporate long-range spatial interactions. We consider a generalization of the model discussed
in~\cite{dipierro2020disconnectedness}, a minimal surface confined within a cylinder with external
data separated by a slab. \\ 

We demonstrate that for any external data separated by a slab containing symmetric parts of the
cylinder, the minimal surface exhibits similar behavior as shown
in~\cite{dipierro2020disconnectedness}. Our results indicate that when the slab is wide, the minimal
surface becomes disconnected from the data, while a narrow slab allows connection. This allows the
behavior of similar models with symmetrically placed data to be predicted. Furthermore, the research
demonstrates that for sufficiently narrow slabs, the surface \enquote{sticks} to the cylinder. \\ 

Finally, we present an example where the minimizer is completely disconnected from the external
data, a phenomenon that is unique to nonlocal minimal surfaces. This work provides valuable insights
into the behavior of these emerging mathematical objects and their interaction with external data.


\section*{Zusammenfassung}

Nichtlokale Minimalflächen, die in einem Zylinder eingeschlossen sind, zeigen ein einzigartiges
Verhalten, das von externen Daten abhängt. Diese Arbeit beschäftigt sich mit diesen Flächen, die im
Vergleich zu klassischen Minimalflächen weitreichende räumliche Wechselwirkungen einbeziehen. Wir
betrachten eine Verallgemeinerung des in~\cite{dipierro2020disconnectedness} diskutierten Modells,
einer in einem Zylinder eingeschlossenen Minimalfläche mit externen Daten, die durch eine Platte
getrennt sind. \\

Wir beweisen, dass für beliebige externe Daten, die durch eine Platte getrennt sind, die
symmetrische Teile des Zylinders enthält, ein ähnliches Verhalten wie
in~\cite{dipierro2020disconnectedness} besitzen. Die Ergebnisse zeigen, dass die Minimalfläche
bei breiter Platte von den Daten getrennt wird, während eine schmale Platte eine Verbindung
ermöglicht. Dies erlaubt uns, das Verhalten ähnlicher Modelle mit symmetrisch platzierten Daten
vorherzusagen. Darüber hinaus zeigen die Untersuchungen, dass die Fläche bei ausreichend schmalen
Platten am Zylinder \enquote{klebt}. \\

Schließlich präsentieren wir ein Beispiel, bei dem die Minimierer vollständig von den externen Daten
getrennt sind, ein Phänomen, das nur bei nichtlokalen Minimalflächen auftritt. Diese Arbeit liefert
wertvolle Erkenntnisse über das Verhalten dieser neuen mathematischen Objekte und ihre
Wechselwirkung mit externen Daten.

