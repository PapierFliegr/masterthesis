\chapter{Abstract}
\label{ch:abstract}

\begin{TODO}[GENERAL STUFF]~\\
    1. Clean up Bibliography\\
    2. Reduce number of tagged equations\\
    3. Remove space between \( \partial A \)?\\
    4. Complete List of symbols\\
    5. Edit colors to fit TUMColors\\
    6. Edit figures to smaller and increase font\\
    7. 
\end{TODO}

\section*{Abstract}
Nonlocal minimal surfaces confined within a cylinder exhibit unique behaviors dependent on
external data. This thesis delves into these surfaces, which incorporate long-range
spatial interactions compared to classical minimal surfaces. We consider two variations of
the model discussed in~\cite{dipierro2020disconnectedness}, a minimal surfaces confined
within a cylinder.\newline

We investigate two scenarios: varying the height and width of data outside a separating
slab. The results show that when the slab is wide, the minimal surface becomes
disconnected from the data, while a narrow slab allows connection. This allows us to
predict the behavior of similar models with symmetrically placed data. Additionally, the
research reveals that for sufficiently narrow slabs, the surface \enquote{sticks} to the
cylinder.\newline 

Finally, we present an example where the minimizer is completely disconnected from the
external data, a phenomenon unique to nonlocal minimal surfaces. This work provides
valuable insights into the behavior of these emerging mathematical objects and their
interaction with external data.


\section*{Zusammenfassung}
In Zylindern eingeschlossene nichtlokale Minimalflächen zeigen ein einzigartiges
Verhalten, das von externen Daten abhängt. Diese Arbeit befasst sich mit diesen Flächen,
die im Vergleich zu klassischen Minimalflächen weitreichende räumliche Wechselwirkungen
berücksichtigen. Wir betrachten zwei Varianten des in~\cite{dipierro2020disconnectedness}
diskutierten Modells, einer in einem Zylinder eingeschlossenen Minimalfläche.\newline

Dabei untersuchen wir zwei Szenarien: die Variation der Höhe und der Breite von Daten
außerhalb einer trennenden Platte. Die Ergebnisse zeigen, dass die Minimalfläche bei
breiter Platte von den Daten getrennt wird, während eine schmale Platte eine Verbindung
ermöglicht. Dies erlaubt uns, das Verhalten ähnlicher Modelle mit symmetrisch angeordneten
Daten vorherzusagen. Darüber hinaus zeigt die Forschung, dass die Fläche bei ausreichend
schmalen Platten am Zylinder \enquote{haftet}.\newline

Schließlich präsentieren wir ein Beispiel, bei dem der Minimierer vollständig von den
externen Daten getrennt ist, ein Phänomen, das für nichtlokale Minimalflächen einzigartig
ist. Diese Arbeit liefert wertvolle Erkenntnisse über das Verhalten dieser neuen
mathematischen Objekte und ihre Wechselwirkung mit externen Daten.