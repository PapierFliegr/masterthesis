% Conclusion
\chapter{Conclusion}
\label{ch:conclusion}

After starting with a short introduction to the classical minimal surfaces theory, we introduced
the recent concept of nonlocal minimal surfaces. We presented the fractional perimeter, a generalization
of the classical perimeter incorporating long-range correlations. We considered a generalization of
the model considered in~\cite{dipierro2020disconnectedness} and showed that the minimizer exhibits
similar behavior depending on the width of the slab. \\ 

In the last part of the thesis, we gave an example of a model where the external data and prescribed
set are disconnected, but the minimizer is not the external data itself. This is a property unique
to nonlocal minimizers. \\ 

While a lot of properties of nonlocal minimizers have been studied, there are still many open
questions. Here we showed that for models such as considered in \Cref{ch:models}, the minimizer
changes topology depending on the width of the slab, but we don't know when and how this happens.
A promising approach would be a barrier construction as shown in~\cite{dipierro2012asymptotics}.
Another interesting question is whether there exists \( s_{0}  \) such that the minimizer is not
the external data for all models such that the external data is unbounded and the distance between
the external data and prescribed set is nonzero.
