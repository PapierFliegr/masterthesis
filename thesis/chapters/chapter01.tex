\chapter{Introduction}

To use the \LaTeX{} templates provided here you will need to add the directory \verb|tum-templates| as a local package directory to your \LaTeX{} distribution. An easy way to do this is by setting the environment variable \verb|TEXINPUTS| to \verb|.//:| on Linux/Mac systems and to \verb|.//;| on a windows machine (meaning: search the current directory and its subdirectories for packages first, then use the usual search path). On a Linux or Mac you can compile this document to a PDF file in a terminal through the following commands (the first command needs to be issued only once):
\begin{verbatim}
export TEXINPUTS=.//:
pdflatex master
bibtex master
pdflatex master
\end{verbatim}

On a windows computer, you would use the following commands in a terminal:

\begin{verbatim}
set TEXINPUTS=.//;
pdflatex master
bibtex master
pdflatex master
\end{verbatim}


\section{First Section of the Introduction}%
\label{sec:first-sect-intr}
Hier folgt eine ausführliche Erklärung und Motivation. Insbesondere weisen wir auf den wunderbaren Artikel von \textcite{Edmonds:1965} und auf~\cite{GareyJohnson:1979} für weitere Hintergründe.

\section{Second Section of the Introduction}%
\label{sec:second-sect-intr}

Wichtige Informationen finden sich in \cref{tab:wonderful-table}.

\begin{table}[hbt]
  \centering
  \begin{tabular}{rl}
    \toprule%
    \textbf{Name}& \textbf{Place of Birth}\\ \midrule
    Gauß & Braunschweig\\
    Euler & Basel\\
    Edmonds & Washington, D.\@C.\@\\
    \bottomrule
  \end{tabular}

  \caption{A most wonderful table}%
  \label{tab:wonderful-table}
\end{table}

\subsection{A Lonesome Subsection}%
\label{sec:lonesome-subsection}
Eine ausführliche \enquote{Erklärung} findet der aufmerksame Leser in \cref{sec:first-sect-intr}.

\section{Figure of a graph}

huhfiusdbf iuhfui sdhfuihsduf sdiuhsd fiusdhf dsfiusdhf suisdfh dshiusdhf sdiuhsdf uidiu
fdsf sdfsdknf fds oihfiwuehf udshfuidshf uidhf usdhf dshfisdufh hfds fiusdhf uihfu hsuifh
iusdhf uisdhf sdhuifhsdiuhfusdhf uhiufhuisdhf uihsduifh suihfusdhfuh iushdfuihsd
uifhsduifsd fhsdiuf hsduifh uisdhuihsuidhfiu shfuihsdiu fhsdiufh sdifhsdiu fuisdhf hiu
sduif sduifh dsfuidshf sdiufh iusdfhiusd fisudfh dsufihsdiuf sduifhsdui fhdsuifhsdui
fsduifh sdiufh sdiufhsdiufh 

\begin{figure}[h] 
    \centering
    \input{graphs/graph01.tex}
    \caption{Test}
\end{figure}

huhfiusdbf iuhfui sdhfuihsduf sdiuhsd fiusdhf dsfiusdhf suisdfh dshiusdhf sdiuhsdf uidiu
fdsf sdfsdknf fds oihfiwuehf udshfuidshf uidhf usdhf dshfisdufh hfds fiusdhf uihfu hsuifh
iusdhf uisdhf sdhuifhsdiuhfusdhf uhiufhuisdhf uihsduifh suihfusdhfuh iushdfuihsd
uifhsduifsd fhsdiuf hsduifh uisdhuihsuidhfiu shfuihsdiu fhsdiufh sdifhsdiu fuisdhf hiu
sduif sduifh dsfuidshf sdiufh iusdfhiusd fisudfh dsufihsdiuf sduifhsdui fhdsuifhsdui
fsduifh sdiufh sdiufhsdiufh 

\clearpage{}

Hier geht es weiter mit dem Text.